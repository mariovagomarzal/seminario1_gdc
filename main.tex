\documentclass[a4paper, 11pt]{article}

%==========%
% Preamble %
%==========%

%=== Language ===%
\usepackage[catalan]{babel}

%=== 'geometry' package ===%
\usepackage{geometry}

%=== Colors ===%
\usepackage[svgnames]{xcolor}
\newcommand{\sectioncolor}{\color{purple}}

%=== 'hyperref' package ===%
\usepackage[pdfusetitle]{hyperref}
\hypersetup{
  colorlinks=true,
}

%=== 'graphicx' package ===%
\usepackage{graphicx}

%=== 'enumitem' package ===%
\usepackage{enumitem}

%=== Document title ===%
\usepackage{titling}

\pretitle{\begin{center}\huge\sffamily\bfseries}
\posttitle{\end{center}}

\preauthor{\begin{center}\Large\sffamily\begin{tabular}[t]{c}}
\postauthor{\end{tabular}\end{center}}

\predate{\begin{center}\Large\sffamily}
\postdate{\end{center}}

%=== Section titles ===%
\usepackage{titlesec}

\titleformat{\section}{\Large\sffamily\bfseries}
  {\sectioncolor{}{\normalfont\sffamily\S}\thesection}
  {1ex}{}

\titleformat{\subsection}{\large\sffamily\bfseries}
  {\sectioncolor{}\thesubsection}
  {1ex}{}

\titleformat{\subsubsection}{\sffamily\bfseries}
  {\sectioncolor{}\thesubsubsection}
  {1ex}{}

%=== Header and footer ===%
\usepackage{fancyhdr}
\pagestyle{fancy}

\setlength{\headheight}{14pt}
\addtolength{\topmargin}{-2pt}

\fancyhf{}
\fancyhead[L]{\sffamily\bfseries Seminari 1 de GDC}
\fancyhead[R]{\sffamily\bfseries\thepage}
\fancypagestyle{plain}{
  \fancyhf{}
  \renewcommand{\headrulewidth}{0pt}
  \fancyfoot[C]{\sffamily\thepage}
}

%=== Mathematics ===%
\usepackage{mathtools}
\mathtoolsset{centercolon}

\usepackage{amssymb}

%== Math environments ==%
\usepackage{amsthm}
\usepackage{thmtools}

\declaretheoremstyle[
  headfont=\sffamily\bfseries\color{MediumBlue},
  headpunct={},
  postheadspace=1em,
  spaceabove=1em,
  spacebelow=1em,
]{thmblue}
\declaretheoremstyle[
  headfont=\sffamily\bfseries\color{FireBrick},
  headpunct={:},
  spaceabove=1em,
  spacebelow=1em,
]{thmred}
\declaretheoremstyle[
  headfont=\sffamily\bfseries\color{ForestGreen},
  headpunct={ ---},
  spaceabove=1em,
  spacebelow=1em,
]{thmgreen}
\declaretheoremstyle[
  headfont=\sffamily\bfseries\color{black},
  spaceabove=1em,
  spacebelow=1em,
]{thmblack}

%= Numbered math environments =%
\declaretheorem[name=Teorema, style=thmblue, numberwithin=section]{theorem}
\declaretheorem[name=Proposición, style=thmblue, sibling=theorem]%
  {proposition}
\declaretheorem[name=Corolario, style=thmblue, sibling=theorem]{corollary}
\declaretheorem[name=Lema, style=thmblue, sibling=theorem]{lemma}
\declaretheorem[name=Conjetura, style=thmblue, sibling=theorem]{conjecture}
\declaretheorem[name=Definición, style=thmred, sibling=theorem]{definition}
\declaretheorem[name=Notación, style=thmred, sibling=theorem]{notation}
\declaretheorem[name=Ejemplo, style=thmgreen, sibling=theorem]{example}
\declaretheorem[name=Nota, style=thmgreen, sibling=theorem]{remark}
\declaretheorem[name=Problema, style=thmblack]%
  {problem}
\declaretheorem[name=Custión, style=thmblack, sibling=problem]%
  {question}
\declaretheorem[name=Ejercicio, style=thmblack, sibling=problem]%
  {exercise}

%= Unnumbered math environments =%
\declaretheorem[name=Teorema, style=thmblue, numbered=no]{theorem*}
\declaretheorem[name=Proposición, style=thmblue, numbered=no]{proposition*}
\declaretheorem[name=Corolario, style=thmblue, numbered=no]{corollary*}
\declaretheorem[name=Lema, style=thmblue, numbered=no]{lemma*}
\declaretheorem[name=Conjetura, style=thmblue, numbered=no]{conjecture*}
\declaretheorem[name=Definición, style=thmred, numbered=no]{definition*}
\declaretheorem[name=Notación, style=thmred, numbered=no]{notation*}
\declaretheorem[name=Ejemplo, style=thmgreen, numbered=no]{example*}
\declaretheorem[name=Nota, style=thmgreen, numbered=no]{remark*}
\declaretheorem[name=Problema, style=thmblack, numbered=no]{problem*}
\declaretheorem[name=Custión, style=thmblack, numbered=no]{question*}
\declaretheorem[name=Exercici, style=thmblack, numbered=no]{exercise*}

\newenvironment{solution}%
  {\begin{proof}[Solució] \color{MidnightBlue!80!black}}%
  {\end{proof}}

\input{tex/macros.tex}

\title{Seminari 1 de Geometria diferencial clàssica}
\author{Ana Martínez \and Javier Orejuela \and%
  Pedro Pasalodos Guiral \and Mario Vago Marzal}
\date{Curso 2023--2024}

\begin{document}
  \maketitle

  \begin{exercise}[Examen 2a convocatòria, juny 2023]
    Considerem la corba parametritzada
    \[
      \alpha(t) = (e^{at} \cos t, e^{at} \sin t)
    \]
    amb $a > 0$. Siga $T_t$ el triangle que té per vèrtexs el punt $P =
    \alpha(t)$ y les interseccions de les rectes tangent i normal
    d'$\alpha$ en $P$ amb la recta que pasa por l'origen de coordenades y
    és perpendicular al vector de posició de $P$. Prova que l'àrea $A(t)$
    de $T_t$ és inversament proporcional al quadrat de la curvatura
    d'$\alpha$ en $t$.
  \end{exercise}

  \begin{solution}
    Notem que podem expressar la corba $\alpha$ amb complexos en forma
    exponencial com
    \[
      \alpha(t) = e^{at + ti} = e^{(a + i)t},
    \]
    fet el qual facilitarà els càlculs.

    En primer lloc, calcularem la curvatura d'$\alpha$. Així, notem que
    \[
      \alpha'(t) = (a + i)e^{(a + i)t}
    \]
    i que
    \[
      \alpha''(t) = (a + i)^2e^{(a + i)t}.
    \]

    Pel atra banda, tenim que
    \[
      J \alpha'(t) = i (a + i)e^{a + i}t =
      (ai - 1)e^{(a + i)t},
    \]
    amb el que
    \[
      \begin{split}
        \gen{\alpha''(t), J \alpha'(t)}
        &= \re\left[(a + i)^2e^{(a + i)t}
        (-ai - 1)e^{(a + i)t}\right] \\
        &= -\re\left[e^{2a}(a^3i -a^2 -ai - 1)\right] \\
        &= e^{2a} (a^2 + 1).
      \end{split} 
    \]

    Per últim, per a calcular la norma, tenim que
    \[
      \norm{\alpha'(t)}^2 
      = \gen{(a + i)e^{(a + i)t}, (a - i)e^{(a - i)t}}
      = (a^2 + 1)e^{2at},
    \]
    és a dir,
    \[
      \norm{\alpha'(t)} = \sqrt{a^2 + 1}e^{at}.
    \]

    Així, podem calcular la curvatura d'$\alpha$ com
    \[
      \kappa(t) = \frac{\gen{\alpha''(t), J \alpha'(t)}}%
      {\norm{\alpha'(t)}^3}
      = \frac{e^{2a}(a^2 + 1)}{(\sqrt{a^2 + 1}e^{at})^3}
      = \frac{e^{-at}}{\sqrt{a^2 + 1}}.
    \]
    
    Ara, per a calcular l'àrea del triangle $T_t$, primer hem de trobar
    les rectes que ho formen. A continuació, trobem un esquema de la
    situació:
    \begin{center}
      \begin{tikzpicture}
        \node (P) at (0, 0) {};
        \draw[fill] (P) circle (1pt);
        \node[above right] at (P) {$P = \alpha(t)$};

        \node (N) at (0, -3) {};
        \draw[fill] (N) circle (1pt);
        \node[below] at (N) {$N$};

        \node (M) at (-2, 0) {};
        \draw[fill] (M) circle (1pt);
        \node[left] at (M) {$M$};

        \draw[->, thick] (P.center) -- (0, -1) node[midway, right] {$\vec{t}$};
        \draw (P.center) -- (N.center)
          node[midway, right] {$r_2$};

        \draw[->, thick] (P.center) -- (-1, 0) node[midway, above]
        {$\vec{n}$};
        \draw (P.center) -- (M.center)
          node[midway, above] {$r_3$};

        \draw (M.center) -- (N.center)
          node[midway, left] {$r_1$};

        \node (O) at (190:1.8) {};
        \draw[fill] (O) circle (1pt);
        \node[below left] at (O) {$O$};
        \draw[dashed] (O.center) -- (P.center);
      \end{tikzpicture}
    \end{center}
    
    Així, tenim que
    \[
      r_1: 0 + J \alpha(t)\lambda_1 = i e^{at + ti} \lambda_1,
    \]
    amb $\lambda_1 \in \RR$, que
    \[
      r_2: \alpha(t) + \alpha'(t) \lambda_2 =
      e^{at + ti} + (a + i)e^{at + ti} \lambda_2,
    \]
    amb $\lambda_2 \in \RR$ i que
    \[
      r_3: \alpha(t) + J \alpha'(t) \lambda_3 =
      e^{at + ti} + i(a + i)e^{at + ti} \lambda_3,
    \]
    amb $\lambda_3 \in \RR$.

    Per a trobar $N$, volem $r_1 \cap r_2$. Si expressem les rectes en 
    $\RR^2$, tenim que
    \[
      r_1: (a_1, b_1) + (c_1, d_1) \lambda_1
    \]
    i que
    \[
      r_2: (a_2, b_2) + (c_2, d_2) \lambda_2,
    \]
    on, per l'expansió
    \[
      i e^{at + ti} \lambda_1 
      = e^{at + ti} + (a + i)e^{at + ti} \lambda_2,
    \]
    tenim que
    \[
      [\underbrace{-\sin t}_{c_1}
      + i\underbrace{\cos t}_{d_1}
      ] \lambda_1
      = (\underbrace{\cos t}_{a_2}
      + i\underbrace{\sin t}_{b_2})
      + [\underbrace{a\cos t - \sin t}_{c_2}
      + i (\underbrace{a\sin t + \cos t}_{d_2})
      ] \lambda_2,
    \]

    de manera que el sistema
    \[
      \begin{cases}
        a_1 + c_1 \lambda_1 = a_2 + c_2 \lambda_2 \\
        b_1 + d_1 \lambda_1 = b_2 + d_2 \lambda_2
      \end{cases}
      \iff
      \begin{cases}
        -\sin (t) \lambda_1 = \cos (t) +
        \left[a\cos (t) - \sin (t)\right] \lambda_2 \\
        \cos (t) \lambda_1 = \sin (t) +
        \left[\sin (t) + \cos (t)\right] \lambda_2
      \end{cases}
    \]
    té solució $(\lambda_1, \lambda_2) = (-\frac{1}{a}, -\frac{1}{a})$.

    Anàlogament, per a trobar $M$, volem $r_1 \cap r_3$. De nou, de
    l'expansió
    \[
      i e^{at + ti} \lambda_1 
      = e^{at + ti} + i(a + i)e^{at + ti} \lambda_3,
    \]
    tenim que
    \[
      [\underbrace{-\sin t}_{c_1}
      + i\underbrace{\cos t}_{d_1}
      ] \lambda_1
      = (\underbrace{\cos t}_{a_3}
      + i\underbrace{\sin t}_{b_3})
      + i[\underbrace{a\cos t - \sin t}_{c_3}
      + i (\underbrace{a\sin t + \cos t}_{d_3})
      ] \lambda_3,
    \]
    de manera que el sistema
    \[
      \begin{cases}
        a_1 + c_1 \lambda_1 = a_3 + c_3 \lambda_3 \\
        b_1 + d_1 \lambda_1 = b_3 + d_3 \lambda_3
      \end{cases}
      \iff
      \begin{cases}
        -\sin (t) \lambda_1 = \cos (t) +
        \left[a\cos (t) - \sin (t)\right] \lambda_3 \\
        \cos (t) \lambda_1 = \sin (t) +
        \left[\sin (t) + \cos (t)\right] \lambda_3
      \end{cases}
    \]
    té solució $(\lambda_1, \lambda_3) = (a, 1)$.

    Ara calcularem els costats base ($\vec{PM}$) i altura ($\vec{PN}$) del triangle:
    \[
      M = r_1(a) = ai e^{(a + i)t} \quad \text{i} \quad
      N = r_1\left(-\frac{1}{a}\right) = -\frac{e^{(a + i)t}}{a}.
    \]

    Calculem la longitud dels costats $\vec{PM}$ i $\vec{PN}$ on
    \[
      \vec{PM} = ae^{(a + i)t}i - e^{(a + i)t}
      \quad \text{i} \quad
      \vec{PN} = -\frac{e^{(a + i)t}}{a}i - e^{(a + i)t}.
    \]
    Així, les seues longituds són
    \[
      \norm{\vec{PM}} = e^{at} \sqrt{1 + a^2}
      \quad \text{i} \quad
      \norm{\vec{PN}} = \frac{e^{at} \sqrt{1 + a^2}}{a}.
    \]
    
    Per tant, l'àrea del triangle $T_t$ és
    \[
      A(t) = \frac{1}{2} \norm{\vec{PM}} \norm{\vec{PN}}
      = \frac{1}{2a} e^{2at} (1 + a^2).
    \]

    Concloem que
    \[
      A(t) = \frac{1}{2a} \cdot \frac{1}{\kappa(t)^2},
    \]
    com volíem demostrar.
  \end{solution}

  \begin{exercise}[Primer parcial, gener 2023]
    Siguen $O$ el punt $(0, 0)$, $r_0$ la recta d'equació $x=1$ i $a \in
    \RR^+$. Es considera per a cada $t \in \RR$ la corba la recta de
    pendent $t$ que passa per l'origen $O$. Aquesta recta talla a $r_0$ en
    un punt $P(t)$. Siga $Q(t)$ el punt de la recta $r_1$ a distància $a$
    de $P(t)$ situat al mateix semipla que $O$ respecte a $r_0$.

    \begin{enumerate}[label=(\alph*)]
      \item Dona una parametrització $\alpha_a$ dels punts $Q(t)$ i troba
      per a quins valors d'$a$ la corba parametritzada $\alpha_a$ és
      regular.
      \item Calcula la curvatura amb signe d'$\alpha_a$ i troba per a quins
      valors d'$a$ és diferent de $0$.
    \end{enumerate}
  \end{exercise}

  \begin{solution}
    \begin{enumerate}[label=(\alph*)]
      \item Dona una parametrització $\alpha_a$ dels punts $Q(t)$ i troba per a quins valors de $a$ la corba
      parametritzada $\alpha_a$ és regular.
    
      Com podem adonar-nos en el dibuix, el vector director de la recta és 
      $\vec{d} = (1,t)$. Si el normalitzem, obtenim $\hat{d} = \frac{1}{\sqrt{1+t^2}}(1,t)$.
    
      Així, com volem trobar el punt sobre aquesta recta que está a distància $\sqrt{1+t^2}-a$,
      obtenim
    
      $$Q(t) = (\sqrt{1+t^2}-a)\left(\frac{1}{\sqrt{1+t^2}}\right) (1,t) = \left(1-\frac{a}{\sqrt{1+t^2}}\right) (1,t)$$
    
      A l'enunciat ens sugereixen fer el canvi $t=\tan\theta$. D'aquesta manera, tenim que 
      $\sqrt{1+\tan^2 (\theta)} = \frac{1}{\cos \theta}$. Per tant, reescrivint
    
      $$Q (\theta) = (1- a\cos \theta)(1,\tan \theta) = (1-a\cos \theta, \tan \theta - a\sin \theta)$$
    
      Ara, calcularem la primera derivada per comprovar quan la corba és regular.
    
      $$
      Q'(\theta) = (a\sin \theta, \sec^2 \theta - a\cos \theta) 
      $$
    
      Per a vore quan no és regular, volem trobar per a quins valors de $a$ i $\theta$
      tenim que $Q'(\theta) = (0,0)$. Així,
    
      $$
      \begin{cases}
        a\sin\theta &= 0\\
        sec^2\theta - a\cos\theta &= 0
      \end{cases}
      $$
    
      De la primera equació tenim que $a=0$ o $\sin\theta = 0 $.
      
      \begin{itemize}
        \item Si $a=0$, de la segona equació ens queda $\sec^2 \theta = 0 \implies \sec \theta = 0$.
        Així, aquest cas no es pot donar.
      
        \item Si $\sin\theta = 0$, tenim que $\theta=0$ perquè en el canvi
        de variable tenim que $\theta \in ]-\frac{\pi}{2},\frac{\pi}{2}[$.
        Ara, de la segona equació tenim que $1-a=0 \implies a=1$.
      \end{itemize}
    
      Per tant, $Q(\theta)$ no és regular quan $a=1$ i el punt problemàtic es dona quan $\theta=0$.
      Aleshores, $Q(\theta)$ és regular per a tot $a\neq 1$.
    
    
      \item Calcula la curvatura amb signe de $\alpha_a$ i troba per a quins valors de $a$ és diferent de zero.
      
      Primer, calcularem la segona derivada de $Q(\theta)$.
  
      $$Q''(\theta) = (a\cos \theta, 2\sec^2\theta \tan\theta + a\sin\theta)$$
  
      Ara calcularem la norma de $Q'(\theta)$
  
      \begin{align*}
        ||Q'(\theta)||^2 &=a^2\sin^2\theta + (\sec^2\theta- a\cos\theta)^2 =\\
        &= a^2\sin^2\theta + \sec^4\theta + a^2\cos^2\theta - 2a\sec\theta = \\
        &=a^2 + \sec^4 \theta -2a\sec\theta
      \end{align*}
  
      Així, calculem
  
      \begin{align*}
        \gen{Q''(\theta), JQ'(\theta)} &= (a\cos \theta, 2\sec^2\theta \tan\theta + a\sin\theta) \cdot ( -\sec^2 \theta + a\cos \theta, a\sin \theta) =\\
        &= -a\sec^2\theta \cos \theta + a^2 \cos^2\theta + 2a\sin\theta \tan\theta \sec^2\theta + a^2\sin^2\theta = \\
        &= a^2 -a\sec\theta + 2a\tan^2\theta \sec\theta = \\
        &= a^2 -a\sec\theta + 2a \sin^2\theta \sec^3\theta
      \end{align*}
  
      Per tant, tenim
  
      $$\kappa_Q (\theta) = \frac{\gen{Q''(\theta), JQ'(\theta)}}{||Q'(\theta)||} 
      = \frac{a^2 -a\sec\theta + 2a \sin^2\theta \sec^3\theta}{(a^2 + \sec^4 \theta -2a\sec\theta)^{\frac{3}{2}}}$$
  
  
      Comprovarem quan la curvatura és zero.
  
      \begin{align*}
        \kappa_Q (\theta) = 0 &\iff a^2 -a\sec\theta + 2a \sin^2\theta \sec^3\theta = 0\\
        & \iff a(a-\sec\theta +2\sin^2\theta \sec^3\theta)=0
      \end{align*}
  
      Notem que això passa quan $a=0$ o $a-\sec\theta +2\sin^2\theta \sec^3\theta=0$.
      Així, la curvatura és zero o bé quan $a=0$ o bé quan $a = \sec\theta - 2\sin^2\theta \sec^3\theta$,
      per a qualsevol $\theta \in ]-\frac{\pi}{2},\frac{\pi}{2}[$.
    \end{enumerate}    
  \end{solution}

  \begin{exercise}[Primer parical, gener 2018]
    Siga $h: ]0, 2\pi[ \longrightarrow \RR$ una funció diferenciable i siga
    $\alpha ]0, 2\pi[ \longrightarrow \RR^2$ una corba parametritzada
    definida per l'expressió
    \[
      \alpha(t) = (h(t) \cos t - h'(t) \sin t,
      h(t) \sin t + h'(t) \cos t).
    \]

    \begin{enumerate}[label=(\alph*)]
      \item Determina la condició que ha de complir $h$ per tal que
      $\alpha$ siga regular. Per a la resta de l'exercisi se suposorà que
      es compleix aquesta condició.
      \item Calcula la curvatura amb signe de $\alpha$.
      \item Comprova que, per a tot $t \in ]0, 2\pi[$, les rectes tangents
      a $\alpha$ en $t$ i en $t + \pi$ són para\lgem{}eles i determina la
      distància entre elles. A aquesta distància se l'anomena amplada de
      $\alpha$ en $t$.
      \item Se considera la funció $h(t) = a \cos^2 (kt/2) + b$, on $a$ i
      $b$ són nombres reals positius i $k$ és un enter positiu imparell.
      Demostra que en auqest cas, la corba $\alpha$ és d'amplada constant
      (i calcula el seu valor).
      \item Demostra que la corba $\alpha$ de l'apartat anterior té
      curvatura constant si i només si $k = 1$.
    \end{enumerate}
  \end{exercise}

  \begin{solution}
    \begin{enumerate}[label=(\alph*)]
      \item Considerem $\alpha(t)=(h(t) \cos t - h'(t) \sin t, h(t) \sin t + h'(t) \cos t)$.
      Per a comprovar quan $\alpha(t)$ és regular, primer derivarem
      
      \begin{align*}
      \dot{\alpha}(t) &= (\cancel{h'(t)\cos t} - h(t)\sin t -h''(t) \sin t - \cancel{h'(t)\cos t}, \\
      &
                           \cancel{h'(t) \sin t} + h(t)\cos t + h''(t)\cos t - \cancel{h'(t) \sin t})\\
      &= (h(t)+h''(t))(-\sin t, \cos t) = (h(t)+h''(t))(\sin (-t), \cos (-t))\\
      &= (h(t)+h''(t))e^{-it}
      \end{align*}
  
      Així, tenim que $\alpha(t)$ no és regular$\iff \dot{\alpha}(t) = 0 \iff h(t)+h''(t)=0 \iff h(t)=-h''(t)$ 
  
      Per tant, $\alpha(t)$ és regular sempre que $h(t) \neq -h''(t)$ per a tot $t \in ]0, 2\pi[$.
      \item Com que $\alpha$ és regular, podem calcular la curvatura amb signe de $\alpha$ com
      \[
        \begin{split}
          \kappa(t) &= \frac{\gen{\alpha''(t), J \alpha'(t)}}%
        {\norm{\alpha'(t)}^3} \\
        &= \frac{\gen{-i(h(t) + h''(t))e^{-it},%
        i(h(t) + h''(t))e^{-it}}}{(h(t) + h''(t))^3} \\
        &= \frac{1}{h(t) + h''(t)}.
        \end{split}
      \]
      \item En la forma complexa que hem vist al primer apartar, és clar
      que
      \[
        \alpha(t + \pi) = e^{\pi} (h'(t) + h''(t))e^{-it},
      \]
      és a dir, que la recta tangent a $\alpha$ en $t + \pi$ és la mateixa.

      Per a calcular l'amplada, observem el següent esquema:
      \begin{center}
        \begin{tikzpicture}
          \draw (0, 0) -- (5, 0);
          \draw (0, -3) -- (5, -3);
        
          \draw[fill] (2,0) circle (1pt);
          \node (t) at (2,0) {};
          \node[above] at (t) {$\alpha(t)$};
          \draw[fill] (3.5,-3) circle (1pt);
          \node (tpi) at (3.5,-3) {};
          \node[below] at (tpi) {$\alpha(t + \pi)$};
        
          \draw[dashed] (t.center) -- (tpi.center)
            node[midway, right] {$h(t) - h(t + \pi)$};
          \draw (t.center) -- (2,-3);
        \end{tikzpicture}
      \end{center}

      És a dir, l'amplada és
      \[
        \gen{\alpha(t + \pi) - \alpha(t), J \alpha(t)} =
        \sqrt{(h(t) + h(t + \pi))^2 + (h'(t) + h'(t + \pi))^2}.
      \]
      \item
      \item Si prenem l'$h$ de l'apartat anterior, tenim que
      \[
        \begin{split}
          \kappa(t) = \frac{1}{h(t) + h''(t)}
          &= \frac{1}{k^2 - 1}\cos^2 \left(\frac{kt}{2}\right)
          - \frac{k^2}{2} - \frac{b}{a}.
        \end{split}
      \]
      
      Així, notem que $\kappa(t)$ es constant si i només si
      \[
        (k^2 - 1) \cos^2 \frac{kt}{2}
      \]
      és constant, és a dir, si i només si $k = \pm 1$ o $k = 0$. No obstant, com
      $k$ és un imparell positiu, necessàriament $k = 1$, el que volíem.
    \end{enumerate}
  \end{solution}
\end{document}
