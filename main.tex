\documentclass[a4paper, 11pt]{article}

%==========%
% Preamble %
%==========%

%=== Language ===%
\usepackage[catalan]{babel}

%=== 'geometry' package ===%
\usepackage{geometry}

%=== Colors ===%
\usepackage[svgnames]{xcolor}
\newcommand{\sectioncolor}{\color{purple}}

%=== 'hyperref' package ===%
\usepackage[pdfusetitle]{hyperref}
\hypersetup{
  colorlinks=true,
}

%=== 'graphicx' package ===%
\usepackage{graphicx}

%=== 'enumitem' package ===%
\usepackage{enumitem}

%=== Document title ===%
\usepackage{titling}

\pretitle{\begin{center}\huge\sffamily\bfseries}
\posttitle{\end{center}}

\preauthor{\begin{center}\Large\sffamily\begin{tabular}[t]{c}}
\postauthor{\end{tabular}\end{center}}

\predate{\begin{center}\Large\sffamily}
\postdate{\end{center}}

%=== Section titles ===%
\usepackage{titlesec}

\titleformat{\section}{\Large\sffamily\bfseries}
  {\sectioncolor{}{\normalfont\sffamily\S}\thesection}
  {1ex}{}

\titleformat{\subsection}{\large\sffamily\bfseries}
  {\sectioncolor{}\thesubsection}
  {1ex}{}

\titleformat{\subsubsection}{\sffamily\bfseries}
  {\sectioncolor{}\thesubsubsection}
  {1ex}{}

%=== Header and footer ===%
\usepackage{fancyhdr}
\pagestyle{fancy}

\setlength{\headheight}{14pt}
\addtolength{\topmargin}{-2pt}

\fancyhf{}
\fancyhead[L]{\sffamily\bfseries Seminari 1 de GDC}
\fancyhead[R]{\sffamily\bfseries\thepage}
\fancypagestyle{plain}{
  \fancyhf{}
  \renewcommand{\headrulewidth}{0pt}
  \fancyfoot[C]{\sffamily\thepage}
}

%=== Mathematics ===%
\usepackage{mathtools}
\mathtoolsset{centercolon}

\usepackage{amssymb}

%== Math environments ==%
\usepackage{amsthm}
\usepackage{thmtools}

\declaretheoremstyle[
  headfont=\sffamily\bfseries\color{MediumBlue},
  headpunct={},
  postheadspace=1em,
  spaceabove=1em,
  spacebelow=1em,
]{thmblue}
\declaretheoremstyle[
  headfont=\sffamily\bfseries\color{FireBrick},
  headpunct={:},
  spaceabove=1em,
  spacebelow=1em,
]{thmred}
\declaretheoremstyle[
  headfont=\sffamily\bfseries\color{ForestGreen},
  headpunct={ ---},
  spaceabove=1em,
  spacebelow=1em,
]{thmgreen}
\declaretheoremstyle[
  headfont=\sffamily\bfseries\color{black},
  spaceabove=1em,
  spacebelow=1em,
]{thmblack}

%= Numbered math environments =%
\declaretheorem[name=Teorema, style=thmblue, numberwithin=section]{theorem}
\declaretheorem[name=Proposición, style=thmblue, sibling=theorem]%
  {proposition}
\declaretheorem[name=Corolario, style=thmblue, sibling=theorem]{corollary}
\declaretheorem[name=Lema, style=thmblue, sibling=theorem]{lemma}
\declaretheorem[name=Conjetura, style=thmblue, sibling=theorem]{conjecture}
\declaretheorem[name=Definición, style=thmred, sibling=theorem]{definition}
\declaretheorem[name=Notación, style=thmred, sibling=theorem]{notation}
\declaretheorem[name=Ejemplo, style=thmgreen, sibling=theorem]{example}
\declaretheorem[name=Nota, style=thmgreen, sibling=theorem]{remark}
\declaretheorem[name=Problema, style=thmblack]%
  {problem}
\declaretheorem[name=Custión, style=thmblack, sibling=problem]%
  {question}
\declaretheorem[name=Ejercicio, style=thmblack, sibling=problem]%
  {exercise}

%= Unnumbered math environments =%
\declaretheorem[name=Teorema, style=thmblue, numbered=no]{theorem*}
\declaretheorem[name=Proposición, style=thmblue, numbered=no]{proposition*}
\declaretheorem[name=Corolario, style=thmblue, numbered=no]{corollary*}
\declaretheorem[name=Lema, style=thmblue, numbered=no]{lemma*}
\declaretheorem[name=Conjetura, style=thmblue, numbered=no]{conjecture*}
\declaretheorem[name=Definición, style=thmred, numbered=no]{definition*}
\declaretheorem[name=Notación, style=thmred, numbered=no]{notation*}
\declaretheorem[name=Ejemplo, style=thmgreen, numbered=no]{example*}
\declaretheorem[name=Nota, style=thmgreen, numbered=no]{remark*}
\declaretheorem[name=Problema, style=thmblack, numbered=no]{problem*}
\declaretheorem[name=Custión, style=thmblack, numbered=no]{question*}
\declaretheorem[name=Exercici, style=thmblack, numbered=no]{exercise*}

\newenvironment{solution}%
  {\begin{proof}[Solució] \color{MidnightBlue!80!black}}%
  {\end{proof}}

\input{tex/macros.tex}

\title{Seminari 1 de Geometria diferencial clàssica}
\author{Ana Martínez \and Javier Orejuela \and%
  Pedro Pasalodos Guiral \and Mario Vago Marzal}
\date{Curso 2023--2024}

\begin{document}
  \maketitle

  \begin{exercise}[Examen 2a convocatòria, juny 2023]
    Considerem la corba parametritzada
    \[
      \alpha(t) = (e^{at} \cos t, e^{at} \sin t)
    \]
    amb $a > 0$. Siga $T_t$ el triangle que té per vèrtexs el punt $P =
    \alpha(t)$ y les interseccions de les rectes tangent i normal
    d'$\alpha$ en $P$ amb la recta que pasa por l'origen de coordenades y
    és perpendicular al vector de posició de $P$. Prova que l'àrea $A(t)$
    de $T_t$ és inversament proporcional al quadrat de la curvatura
    d'$\alpha$ en $t$.
  \end{exercise}

  \begin{solution}
    Notem que podem expressar la corba $\alpha$ amb complexos en forma
    exponencial com
    \[
      \alpha(t) = e^{at + ti} = e^{(a + i)t},
    \]
    fet el qual facilitarà els càlculs.

    En primer lloc, calcularem la curvatura d'$\alpha$. Així, notem que
    \[
      \alpha'(t) = (a + i)e^{(a + i)t}
    \]
    i que
    \[
      \alpha''(t) = (a + i)^2e^{(a + i)t}.
    \]

    Pel atra banda, tenim que
    \[
      J \alpha'(t) = i (a + i)e^{a + i}t =
      (ai - 1)e^{(a + i)t},
    \]
    amb el que
    \[
      \begin{split}
        \gen{\alpha''(t), J \alpha'(t)}
        &= \re\left[(a + i)^2e^{(a + i)t}
        (-ai - 1)e^{(a + i)t}\right] \\
        &= -\re\left[e^{2a}(a^3i -a^2 -ai - 1)\right] \\
        &= e^{2a} (a^2 + 1).
      \end{split} 
    \]

    Per últim, tenim que
    \[
      \norm{\alpha'(t)}^2 = \norm{(a + i)e^{(a + i)t}}^2 =
      % TODO
    \]

    Així, podem calcular la curvatura d'$\alpha$ com
    \[
      \kappa(t) = \frac{\gen{\alpha''(t), J \alpha'(t)}}%
      {\norm{\alpha'(t)}^3}
      = \frac{2^{2a}(a^2 + 1)}{(\sqrt{a^2 + 1}e^{at})^3}
      = \frac{e^{-at}}{\sqrt{a^2 + 1}}.
    \]
    
    Ara, per a calcular l'àrea del triangle $T_t$, primer hem de trobar
    les rectes que ho formen. Així, tenim que
    \[
      r_1: 0 + J \alpha(t)\lambda_1 = i e^{(a+i)t} \lambda_1,
    \]
    amb $\lambda_1 \in \RR$ i que
    \[
      r_2: \alpha(t) + \alpha'(t) \lambda_2 =
      e^{at + ti} + (a + i)e^{at + ti} \lambda_2
    \]
    amb $\lambda_2 \in \RR$.

    Si expressem les rectes en $\RR^2$, tenim que
    \[
      r_1: (a_1, b_1) + (c_1, d_1) \lambda_1
    \]
    i que
    \[
      r_2: (a_2, b_2) + (c_2, d_2) \lambda_2,
    \]
    de manera que el sistema
    \[
      \begin{cases}
        a_1 + c_1 \lambda_1 = a_2 + c_2 \lambda_2 \\
        b_1 + d_1 \lambda_1 = b_2 + d_2 \lambda_2
      \end{cases}
      \iff
      \begin{cases}
        -\sin (t) \lambda_1 = \cos (t) +
        \left[-a\sin (t) - \cos (t)\right] \lambda_2 \\
        \cos (t) \lambda_1 = \sin (t) +
        \left[a\cos (t) - \sin (t)\right] \lambda_2
      \end{cases}
    \]
    té solució $(\lambda_1, \lambda_2) = (a, 1)$.

    Ara calcularem els costats base ($M$) i altura ($N$) del triangle:
    \[
      M = r_1(a) = ai e^{(a + i)t} \quad \text{i} \quad
      r_1(-\frac{1}{a}) = -\frac{e^{(a + i)t}}{a}.
    \]

    Calculem la longitud dels costats $\vec{PM}$ i $\vec{PN}$ on
    \[
      \vec{PM} = ae^{(a + i)t}i - e^{(a + i)t}
      \quad \text{i} \quad
      \vec{PN} = -\frac{e^{(a + i)t}}{a}i - e^{(a + i)t}.
    \]
    Així, les seues longituds són
    \[
      \norm{\vec{PM}} = e^{at} \sqrt{1 + a^2}
      \quad \text{i} \quad
      \norm{\vec{PN}} = \frac{e^{at} \sqrt{1 + a^2}}{a}.
    \]
    
    Per tant, l'àrea del triangle $T_t$ és
    \[
      A(t) = \frac{1}{2} \norm{\vec{PM}} \norm{\vec{PN}}
      = \frac{1}{2a} e^{2at} (1 + a^2).
    \]

    Concloem que
    \[
      A(t) = \frac{1}{2a} \cdot \frac{1}{\kappa(t)^2},
    \]
    com volíem demostrar.
  \end{solution}

  \begin{exercise}[Primer parcial, gener 2023]
    Siguen $O$ el punt $(0, 0)$, $r_0$ la recta d'equació $x=1$ i $a \in
    \RR^+$. Es considera per a cada $t \in \RR$ la corba la recta de
    pendent $t$ que passa per l'origen $O$. Aquesta recta talla a $r_0$ en
    un punt $P(t)$. Siga $Q(t)$ el punt de la recta $r_1$ a distància $a$
    de $P(t)$ situat al mateix semipla que $O$ respecte a $r_0$.

    \begin{enumerate}[label=(\alph*)]
      \item Dona una parametrització $\alpha_a$ dels punts $Q(t)$ i troba
      per a quins valors d'$a$ la corba parametritzada $\alpha_a$ és
      regular.
      \item Calcula la curvatura amb signe d'$\alpha_a$ i troba per a quins
      valors d'$a$ és diferent de $0$.
    \end{enumerate}
  \end{exercise}

  \begin{solution}
    
  \end{solution}

  \begin{exercise}[Primer parical, gener 2018]
    Siga $h: ]0, 2\pi[ \longrightarrow \RR$ una funció diferenciable i siga
    $\alpha ]0, 2\pi[ \longrightarrow \RR^2$ una corba parametritzada
    definida per l'expressió
    \[
      \alpha(t) = (h(t) \cos t - h'(t) \sin t,
      h(t) \sin t + h'(t) \cos t).
    \]

    \begin{enumerate}[label=(\alph*)]
      \item Determina la condició que ha de complir $h$ per tal que
      $\alpha$ siga regular. Per a la resta de l'exercisi se suposorà que
      es compleix aquesta condició.
      \item Calcula la curvatura amb signe de $\alpha$.
      \item Comprova que, per a tot $t \in ]0, 2\pi[$, les rectes tangents
      a $\alpha$ en $t$ i en $t + \pi$ són para\lgem{}eles i determina la
      distància entre elles. A aquesta distància se l'anomena amplada de
      $\alpha$ en $t$.
      \item Se considera la funció $h(t) = a \cos^2 (kt/2) + b$, on $a$ i
      $b$ són nombres reals positius i $k$ és un enter positiu imparell.
      Demostra que en auqest cas, la corba $\alpha$ és d'amplada constant
      (i calcula el seu valor).
      \item Demostra que la corba $\alpha$ de l'apartat anterior té
      curvatura constant si i només si $k = 1$.
    \end{enumerate}
  \end{exercise}

  \begin{solution}
    
  \end{solution}
\end{document}
